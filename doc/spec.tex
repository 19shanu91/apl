\documentclass[11pt]{article}
\usepackage{booktabs}
\usepackage{fullpage}
\usepackage{palatino}
\usepackage{tabularx}
\usepackage{verbatim}


\title{APSIL Language Specification\\
Version 1.0}
\author{\texttt{kmurali@nitc.ac.in} \\  \small {NIT Calicut} }

\begin{document}

\maketitle
\tableofcontents
\pagebreak
\section{Introduction}
\paragraph{}
\textit{APSIL} or \textit{Application Programmer's Simple Integer Language} is a simple and strongly typed programming language. The features and constructs of this language are minimal and mainly intended for testing an experimental operating system. The compiler of APSIL runs on ESIM (\textit{Extended Simple Integer Machine}) architecture.
\paragraph{}
This document describes the syntax, semantics and constructs of APSIL in detail. The structure of APSIL is similar in many aspects to programming languages like C and Java. 

\section{General Program Structure}

A typical APSIL program is orgnaized in the following way. 

\begin{verbatim}
Global Declarations
. 
. 
Function Definitions
.
. 
Main Function
\end{verbatim}

\end{document}
